% Infotheo, not knowing anything and relying on it only to find out (thorugh other people) that it might not be usable
% Infotheo, expanded: the difficult part that made this assignment hard and not solvable is that it is hard working with discrete encodings of continuous constructs such as distributions.
% Expansion. The easy ones like N coins or N-sided coins/die 
% Monty Hall, do end up with the right framework or not

This project was approached as a challenging but definitely solvable problem, with a clear goal in mind of being able to prove the Monty Hall problem and an expected linear path to get there, which would include implementing the small Imp language and probabilistic Hoare logic. As it often happens in academia and software projects this path turned out to be a lot more steep and winding and resulted in the limiting of our scope as well as having to accept lesser solutions to a lot of problems. 

A key factor for how the project went, that is not directly related to the subject matter, is what we consider to be noticeably less supported and comprehensive tooling (compared to mainstream programming languages), documentation and archive of forum posts detailing the same unique problems that we have faced. The \textit{infotheo} and \textit{math-comp} Coq libraries contain no documentation that is useful for novice Coq users, such as ourselves. We undoubtedly appreciate the work that the people behind these libraries do, but we cannot consider it to be easily accessible to beginners in program verification, and it has been a huge barrier in trying to implement this project. %Perhaps it is a symptom of being produced by computer scientists and not people who identify as software developers or engineers. We must stress that we mean not to backtalk these library providers, but we cannot ignore our experience.

These Coq libraries have been a very important part of the implementation, and due to the, to us, unexpectedly steep learning curve, we ended up spending a lot of time trying to simply grasp the concepts and types, which made it difficult to provide much when it came to realising the more complicated problems and solution.
This has made it hard for us to take much ownership over the development of the project, which has continuously pushed our level of understanding to it's limits and beyond. In future projects, we will try to ensure an even more honest approach to what challenges are reasonable for us to, not only take on, but also complete with the intended learning outcomes. 

All theorems, lemmas and programs were intended to be fully proven, with an internal agreement, that some results would have higher priority over others. Per the day of officially handing in the project, only few results are actually proven in Coq by the group, while we have had to rely on the trustfulness of our sources when basing our code off of some of their content. Most importantly is however, that we were able to finalise a proof of the case of flipping two coins. Actually stating and proving a probabilistic problem was always the main motivation for this project. Even though the proven problem ended up being a very simple one, it still serves us as a proof of concept, that more interesting cases could be added including the Monty Hall problem, which is what initially caught our attention and drew us to this topic. 


